\documentclass[notitlepage,a4paper]{article}

\usepackage{xltxtra}
\usepackage{amsmath}
\usepackage{amssymb}
\usepackage{amsthm}
\usepackage{tikz}
\usepackage{unicode-math}
\usepackage{fontspec}
\setmathfont{xits-math.otf}
\usepackage{hyperref}

\newtheorem{proposition}{Proposition}
\newtheorem{corollary}{Corollary}
\newtheorem{lemma}{Lemma}

\makeatletter
\def\namedlabel#1#2{\begingroup
    #2%
    \def\@currentlabel{#2}%
    \phantomsection\label{#1}\endgroup
}
\makeatother

\tikzset{node distance=2cm, auto}

\author{Víctor López Juan}
\title{Chapter 2}

\begin{document}
\maketitle

\begin{enumerate}
  \item[12.]

    {\em Prove proposition \ref{prop:ex}. }

    First, a brief reminder of boolean algebras.
    
    A boolean algebra is a tuple $B = (\vert B \vert, ≤, ∧, ∨, ¬, ⊥, ⊤)$
    where $⊥$ and $⊤$ are distinct, and which fulfils the
    following conditions for every $a$, $b$, $c$ $ ∈ B$.
    
    \begin{description}
      \item[\namedlabel{itm:ba-lt}{BA$≤$}]  $(B, ≤)$ is a poset.
      \item[\namedlabel{itm:ba-bot}{BA$⊥$}] $⊥ ≤ a$
      \item[\namedlabel{itm:ba-top}{BA$⊤$}] $a ≤ ⊤$
      \item[\namedlabel{itm:ba-join}{BA$∨$}]   $a ≤ c$ and $b ≤ c$ iff $a ∨ b ≤ c$
      \item[\namedlabel{itm:ba-meet}{BA$∧$}]   $c ≤ a$ and $c ≤ b$ iff $c ≤ a ∧ b$
      \item[\namedlabel{itm:ba-not-1}{BA$¬1$}] $a ≤ ¬b$ iff $a ∧ b = ⊥$
      \item[\namedlabel{itm:ba-not-2}{BA$¬2$}] $¬¬a = a$
    \end{description}

    The operations $∧$ and $∨$ are called meet and join, respectively.

    Boolean algebras together with their homomorphisms form a category.
    The boolean algebra $\text{\bf 2}$ is the initial algebra, where
    $\vert \text{\bf 2} \vert = \{⊥, ⊤\}$.


    For a boolean algebra $B$, a filter $F \subset B$ is a set such
    that, for every $a$, $b$:

    \begin{itemize}
      \item $a \in F$ and $a ≤ b$ ⇒ $b ∈ F$
      \item $∧(F,F) \subset F$
    \end{itemize}

    $B \subseteq B$ is the trivial filter. 
    The set of non-trivial filters can be ordered by inclusion; a maximal element
    is an ultrafilter.

    Before proving Proposition \ref{prop:ex}, we will derive
    some generally useful properties of boolean algebras from the definition above.
    
     \begin{lemma}\label{lem:bot}
       If $F$ is a filter, $⊥ \in F$ iff $F = B$.
     \end{lemma}
     
     \begin{proof}
           If $F$ is a filter, and $⊥ ∈ F$, for every $x$ in $B$, $⊥ ≤ x$,
           $x ∈ F$; which means $F = B$.

           If $F = B$, then $⊥ \in F$.
     \end{proof}

      \begin{lemma}\label{lem:commutative}
       The operations $∧$, $∨$ are commutative.
     \end{lemma}

     \begin{proof}
       Let $a$, $b$ $∈ B$.
       
       $$b ∧ a ≤ b ∧ a \Rightarrow b ∧ a ≤ a$$

       $$b ∧ a ≤ b ∧ a \Rightarrow b ∧ a ≤ b$$
       
       $$b ∧ a ≤ a, b ∧ a ≤ b \Rightarrow b ∧ a ≤ a ∧ b$$

       By renaming variables,  it also holds that $a ∧ b ≤ b ∧ a$.

       Therefore, $a ∧ b = b ∧ a$.

       The case for $∨$ is analogous:

       $$b ∨ a ≤ b ∨ a \Rightarrow a ≤ b ∨ a$$
       
       $$b ∨ a ≤ b ∨ a \Rightarrow b ≤ b ∨ a$$

       Therefore, $a ∨ b ≤ b ∨ a$.

       By renaming, $a ∨ b = b ∨ a$.
       
     \end{proof}
       

     \begin{lemma}\label{lem:associative}
       The operations $∧$, $∨$ are associative.
     \end{lemma}

     \begin{proof}

       It's suffices to prove that $(a ∧ b) ∧ c ≤ a ∧ (b ∧ c)$.

       By \ref{itm:ba-meet}:

       $$(a ∧ b) ∧ c ≤ a ∧ b ≤ a$$ %1
       $$(a ∧ b) ∧ c ≤ a ∧ b ≤ b$$ %2
       $$(a ∧ b) ∧ c ≤ c$$         %3

       Therefore

       $$(a ∧ b) ∧ c ≤ b ∧ c$$

       Finally:

       $$(a ∧ b) ∧ c ≤ a ∧ (b ∧ c)$$

       By commutativity:

       $$c ∧ (a ∧ b) ≤ (b ∧ c) ∧ a$$

       And, renaming and antisymmetry:

       $$(a ∧ b) ∧ c = a ∧ (b ∧ c)$$

       As in the proof of Lemma \ref{lem:commutative}, the case for $∨$ is
       analogous.
       
     \end{proof}

     \begin{lemma}\label{lem:bot-meet}
       For every $x$, $x ∧ ⊥ = ⊥$.
     \end{lemma}
     \begin{proof}
       By \ref{itm:ba-meet}, $x ∧ ⊥ ≤ ⊥$, and,
       by \ref{itm:ba-bot}, $⊥ ≤ x ∧ ⊥$.

       Therefore, $x ∧ ⊥ = ⊥$.
     \end{proof}

     \begin{lemma}\label{lem:top-meet}
       For every $x$, $x ∧ ⊤ = x$.
     \end{lemma}
     \begin{proof}
       By \ref{itm:ba-meet}, $x ∧ ⊤ ≤ x$.

       By \ref{itm:ba-lt}, $x ≤ x$, and, by \ref{itm:ba-top}, $x ≤ ⊤$.
       By \ref{itm:ba-meet}, $x ≤ x ∧ ⊤$.
     \end{proof}
     
     \begin{lemma}\label{lem:self}
       For every $x$, $x ∧ x = x ∨ x = x$.
     \end{lemma}
     \begin{proof}
       $$ x ≤ x, x ≤ x ⇒ x ≤ x ∧ x$$
       $$ x ∧ x ≤ x ∧ x ⇒ x ∧ x ≤ x$$

       Also,

       $$ x ≤ x, x ≤ x ⇒ x ∨ x ≤ x$$
       $$ x ∨ x ≤ x ∨ x ⇒ x ≤ x ∨ x$$
     \end{proof}

     \begin{lemma}\label{lem:filter-ext}
       Let $B$ be a boolean algebra, and $F$ a proper filter, $F \subsetneq B$.
       
       If $¬x \not \in F$, then there exists a
       filter $F^\prime$ such that $F \subseteq F \cup \{x\} \subset F^\prime$,
       and $F^\prime$ is non-degenerate (i.e. $F^\prime \neq B$).
       
    \end{lemma}
    \begin{proof}
           We can extend $F$ to create a new filter $F^\prime$,
           $F \cup \{x\} \subset F^\prime \subsetneq B$:
           
           Now, define the following sequence:

           $$A_0 = F \cup \{ x ∧ f | f ∈ F \}$$
           $$A_{i+1} = Cl_{∧}(A_i) \cup Cl_{≤}(A_i)$$

           Where:

           $$Cl_{∧}(Z) = \{ a ∧ b  \vert a, b \in Z \}$$
           $$Cl_{≤}(L) = \{ z \vert l \in L, z \in B, b ≤ z \}$$

           And $A$ the limit of the sequence:

           $$A = \bigcup_{i=0}^{\infty} A_i$$

           \begin{description}
             \item[$F \subseteq A$:]

               First, by lemma \ref{lem:top-meet}, $x = x ∧ ⊤ ∈ F^\prime$.

               $F \subseteq F \cup \{x\} \subset A_0 \subset A$
               
             \item[Closure under $∧$:]

               Let $a,\,b \in A$. $a \in A_i$, $b \in A_j$, for some
               $i$, $j$. Therefore,
               $a,b \in A_{\max(i,j)}$, and $a ∧ b \in A_{\max(i,j)+1} \subset A$.

             \item[Closure under $≤$:]
               
               Let $a \in A$. Then, $a \in A_i$ for some $i$. If $b \in B$,
               $a ≤ b$, then $b ∈ Cl_{≤}(A_i) \subset A_{i+1} \subset A$.

             \item[Lower-bounded by $A_0$:]
               
               Every element in $x ∈ A$ is greater than or equal to an element
               $y ∈ A_0$.

               {\em Proof by induction}

               \begin{itemize}
                 \item For every $x \in A_0$, $x ≥ x \in A_0$. 
                 \item Assume that all elements in $A_i$ are lower-bounded
                   by $A_0$.

                   Let $c \in A_{i+1}$. Either:
                   
                   \begin{itemize}
                     \item $c \in Cl_{≤}(A_i)$. Then, for some $b \in A_i$,
                       $b ≤ c$. By the induction hypothesis, $\exists y \in A_0$,
                       $y ≤ b ≤ c$.

                     \item $c \in Cl_{∧}(A_i)$. Then, for some $c = a ∧ b$,
                       $a, b \in A_i$. Then, there exists $a^\prime, b^\prime \in A_0$.
                       $a^\prime ≤ a$, $b^\prime ≤ b$.

                       By \ref{itm:ba-meet}, $a^\prime ∧ b^\prime ≤ a^\prime, b^\prime$,
                       so $a^\prime ∧ b^\prime ≤ a, b$.

                       Therefore, $a^\prime ∧ b^\prime ≤ a ∧ b$.

                       For $a^\prime ∧ b^\prime$, either:

                       \begin{itemize}
                         \item $a^\prime ∧ b^\prime = f_1 ∧ f_2 ∈ F \subset A_0$
                         \item $a^\prime ∧ b^\prime = (x ∧ f_1) ∧ f_2 = x ∧ (f_1 ∧ f_2) ∈ x ∧ F \subset A_0$
                         \item $a^\prime ∧ b^\prime = x ∧ (x ∧ f_2) = (x ∧ x) ∧ f_2 = x ∧ f_2 ∈ x ∧ F \subset A_0$
                         \item $a^\prime ∧ b^\prime = (x ∧ f_1) ∧ (x ∧ f_2) = (x ∧ x) ∧ (f_1 ∧ f_2) = x ∧ (f_1 ∧ f_2) ∈ x ∧ F \subset A_0$
                       \end{itemize}

                       Therefore, $a^\prime ∧ b^\prime = y ∈ A_0$, $y ≤ a ∧ b = c$
                  \end{itemize}

                   In conclusion, if $A_{0}$ lower-bounds $A_{i}$, then
                   it lower-bounds $A_{i+1}$.
               \end{itemize}
           \end{description}

           $A$ is a filter, and all of its elements are greater than or
           equal than an element in $A_0$.
           
           Now we want to check that $A$ is proper; namely, $A ≠ B$.

           Assume $⊥ \in A = B$. Then, there is $y \in A_0$, $y ≤ ⊥$,
           so $⊥ = y \in A_0$. Either:

           \begin{itemize}
             \item $⊥ \in F$: This is impossible, because of lemma \ref{lem:bot} and $F \neq B$.
             \item $⊥ = x ∧ f$ for some $f ∈ F$. Then, by \ref{itm:ba-not-1},
               $f ≤ ¬x$. And, because $F$ is a filter,
               $¬x \in F$, which contradicts our assumptions.
           \end{itemize}

           Therefore, $⊥ \not \in A ≠ B$.

           $F^\prime = A$ is the extension we looked for, and is a proper filter.
           
     \end{proof}
     
     \begin{lemma}\label{lem:neg}
       If $U$ is an ultrafilter, for every $x ∈ B$,
       $x ∈ U$ iff $¬x \not \in U$.
     \end{lemma}
     \begin{proof}

       \begin{description}
         
         \item[⇒] Let $x ∈ U$. If $¬x ∈ U$, then $x ∧ ¬x ∈ U$.
           $x ≤ x$ so $⊥ = x ∧ ¬x ∈ U$. This contradicts lemma \ref{lem:bot}.

         \item[⇐] Assume $¬x \not \in U$,

           {\em Reduction ad absurdum} Suppose that $x \not \in U$. Then,
           $U$ can be extended into a filter $U^\prime$ by lemma \ref{lem:filter-ext},
           $U \subsetneq U \cup \{x\} \subseteq U^\prime \subsetneq B$.
           
           But this contradicts $U$ being a maximal filter.
           
           Therefore, $x \in U$.
            
        \end{description}
    \end{proof}

     \begin{lemma}\label{lem:compl-rev}
       $a ≤ b$ iff $¬b ≤ ¬a$
     \end{lemma}
     \begin{proof}
       Assume $a ≤ b$. Then $a ∧ ¬b = ⊥$, or, equivalently,
       $¬b ∧ ¬(¬a) = ⊥$. Therefore, $¬b ≤ ¬a$.
     \end{proof}

     \begin{lemma}\label{lem:morgan}
       Boolean algebras satisfy DeMorgan's laws; that is, for every
       $a$, $b ∈ B$:

       $$¬(a ∨ b) = ¬a ∧ ¬b$$

       $$¬(a ∧ b) = ¬a ∨ ¬b$$
       
     \end{lemma}

     \begin{proof}
       \begin{description}
         \item[$¬a ∧ ¬b ≤ ¬(a ∨ b)$]

           $$a ∧ ¬a ∧ ¬b = ⊥ ∧ ¬b = ⊥$$
           $$b ∧ ¬a ∧ ¬b = ¬a ∧ ⊥ = ⊥$$

           By definition of boolean algebra:

           $a ∧ (¬a ∧ ¬b) = ⊥ \Rightarrow a ≤ ¬(¬a ∧ ¬b)$
           $b ∧ (¬a ∧ ¬b) = ⊥ \Rightarrow b ≤ ¬(¬a ∧ ¬b)$

           Therefore,

           $$a ∨ b ≤ ¬(¬a ∧ ¬b)$$

           By lemma \ref{lem:compl-rev},

           $$¬a ∧ ¬b ≤ ¬(a ∨ b)$$

         \item[$¬(a ∨ b) ≤ ¬a ∧ ¬b$]

           By definition of boolean algebra:

           $$a ≤ a ∨ b ⇒ ¬(a ∨ b) ≤ ¬a$$
           $$b ≤ a ∨ b ⇒ ¬(a ∨ b) ≤ ¬b$$

           Therefore,

           $$¬(a ∨ b) ≤ ¬a ∧ ¬b$$
       \end{description} 

       Therefore,

       $$¬(a ∨ b) = ¬a ∧ ¬b$$

       In particular, for $c,d ∈ B$.

       $$¬((¬c) ∨ (¬d)) = c ∧ d$$

       which is equivalent to:

       $$¬(c ∧ d) = ¬¬(¬c ∨ ¬d) = ¬c ∨ ¬d$$
   \end{proof}
     
   \begin{proposition}\label{prop:ex}
   For every boolean algebra $B$ there is a biyective correspondence
   between ultrafilters and homomorphisms to the two-element boolean
   algebra.

   In particular,  $h : B → \text{\bf 2}$ is an homomorphism of boolean algebras
   if and only if $U = h^{-1}(⊤)$ is an ultrafilter in $B$.
   
   \end{proposition}

   \begin{proof}
   \begin{enumerate}
     \item Every homomorphism $h : B → \text{\bf 2}$ corresponds to an
       ultrafilter in B.

       Let $U = h^{-1}(⊤)$.

       First, prove that $U$ is a filter.

       \begin{itemize}
         \item
           Let $a ∈ U$. Then $h(a) = ⊤$

           Let $b ∈ B$ such that $a ≤ b$. $h$ is an homomorphism, which
           preserves the ordering therefore $⊤ = h(a) ≤ h(b)$.

           On the other hand, $h(b) ≤ ⊤$ because $\text{\bf 2}$ is a
           boolean algebra. So, $h(b) = ⊤$, and $b ∈ U$.

         \item

           Let $a, b ∈ u$. Then $h(a), h(b) = ⊤$.

           $h$ is an homomorphism; which implies $h(a ∧ b) = h(a) ∧ h(b) = ⊤ ∧ ⊤$.

           For all elements x, $x ≤ ⊤$. Therefore, we have

           $$⊤ ≤ ⊤ ∧ ⊤ ≤ ⊤$$

           Finally, $h(a ∧ b) = ⊤$, so $(a ∧ b) ∈ U$.

      \end{itemize}

      $U$ is non-trivial, as $h(⊥) = ⊥ ≠ ⊤$.

      Now, assume that $U$ is not an ultrafilter. Then, there exists
      some filter $U^\prime$, $U^\prime \subsetneq U$, $U^\prime ≠ B$.
           
      Let $x ∈ U^\prime \setminus U$. Then, $h(¬x) = ¬(h(x)) = ⊤$, so $¬x ∈ U \subset U^\prime$.

      By \ref{itm:ba-not-1}, \ref{itm:ba-not-2} and \ref{lem:bot},
      
      $$x, ¬x ∈ U^\prime ⇒ x ∧ ¬x = ⊥ ∈ U^\prime ⇒ U^\prime = B$$

      … which contradicts $U^\prime$ being non-trivial.

   \item Every ultrafilter $U$ in $B$ corresponds to an homomorphism $h : B → \text{\bf 2}$.

     
     
     Let $h_U : B → \text{\bf 2}$, $h_U(x) = ⊤$ if $x ∈ U$, $h_U(x) = ⊥$ otherwise.

     Now, prove that $h$ is a boolean-algebra homomorphism.

     \begin{description}
       \item[$h$ preserves $⊤$:]

         Let F be a non-empty filter, $x ∈ F$. Then, $⊤ ≥ x$, so
         $⊤ ∈ F$. Therefore, non-empty filters contain $⊤$.

         Suppose that $U$ is empty. Then, by lemma \ref{lem:filter-ext},
         it can be extended to include $⊤$, so it's not maximal.
         
         $U$ is maximal; therefore $U$ is not empty. Let $x ∈ U$.
         By \ref{itm:ba-top}, $x ≤ ⊤$. But $U$ is a filter, so:
         
         $$x ≤ TOP ⇒ ⊤ ∈ U ⇒ h_U(⊤) = ⊤\text{.}$$

       \item[$h$ preserves $⊥$:]

         By lemma \ref{lem:bot}, $⊥ \not \in U$. Therefore, $h(⊥) = ⊥$.

       \item[$h$ preserves $≤$:]

         Let $a, b ∈ B$, $a ≤ b$.

         If $h(a) = ⊥$, then $h(a) = ⊥ ≤ h(b)$.

         If $h(a) = ⊤$, then $a ∈ U$, and, as $a ≤ b$ and $U$ is a
         filter, $b ∈ U$. Therefore:

         $$h(a) = ⊤ ≤ ⊤ = h(b)\text{.}$$

       \item[$h$ preserves $∧$:]

         Let $a, b ∈ B$.

         We want to prove $h(a), h(b) = ⊤$ iff $h(a ∧ b) = ⊤$;
         if any of the values $h(a)$, $h(b)$ is $⊥$, the result should
         be $⊥$.
         
         This equivalent to $a, b ∈ U$ iff $a ∧ b ∈ U$ 

         \begin{description}
           \item[⇒] If $a, b ∈ U$, then $a ∧ b ∈ U$, because $U$ is a filter.
           \item[⇐] $a ∧ b ∈ U$. Assume $a \not \in U$.

             Then, by lemma \ref{lem:neg}, $¬a ∈ U$.

             By lemma \ref{lem:associative} and \ref{lem:bot-meet}.
             
             \begin{align*}
               ¬a ∧ (a ∧ b) = (¬a ∧ a) ∧ b = ⊥ ∧ b = ⊥
             \end{align*}

             But $⊥ \not \in U$, so our assumption is false: $a \in U$.

             The same holds for $b$.
         \end{description}

      
       \item[$h$ preserves $¬$:]

         By lemma \ref{lem:neg}:

         $$h(a) = ⊤ \Rightarrow a ∈ U \Rightarrow ¬a \not \in U \Rightarrow h(¬a) = ⊥ = ¬h(a)$$
         
         $$h(a) = ⊥ \Rightarrow a \not \in U \Rightarrow ¬a ∈ U \Rightarrow h(¬a) = ⊤ = ¬h(a)$$
         
       \item[$h$ preserves $∨$:]

         $h$ preserving joins is equivalent to: 

         $$h(a) = ⊥ \text{ and } h(b) = ⊥ \text{ iff } h(a ∨ b) = ⊥$$

         or, by lemma \ref{lem:neg}:

         $$h(¬a) = ⊤ \text{ and } h(b) = ⊤ \text{ iff } h(¬(a ∨ b)) = ⊤$$
         
         or, by lemma \ref{lem:morgan}:
         
         $$h(¬a) = ⊤ \text{ and } h(¬b) = ⊤ \text{ iff } h((¬a) ∧ (¬b)) = ⊤$$

         … which is true, because $h$ has been shown to preserve meets. 

    \end{description}

  \end{enumerate}

   $h_U^{-1}(⊤) = U$ and $h_{f^{-1}(⊤)} = f$ . Therefore, constructions
   are inverses of each other, and define an isomorphism.

   \end{proof}
  
     
  \item[18.]

    {\em Representing an arbitrary monoid as a set of homomorphisms.}
    
    Let $(\mathbb{N}, +, 0)$ be the free, commutative monoid generated
    by one element $e$.

    $$\mathbb{N} = \{ 0, e, 2·e, 3·e, …, k·e, … \}$$


    Let $(M,·,1)$ be a monoid. For every $m \in M$, define the
    following morphism.

    \begin{eqnarray*}
    h_m : &  \mathbb{N} & →         M   \\
          &        k·e    & \mapsto   m^k \\
    \end{eqnarray*}

    $$h(0) = h(0·e) = m^0 = 1$$ 
    
    $$h_m(k·e + j·e) = h_m((k+j)·e) = m^{(k+j)} = m^k · m^j = h_m(k·e) + h_m(j·e)$$

    Therefore, $h_m$ is a monoid homomorphism.

    Now, let $f : \mathbb{N} → M$ be an arbitrary monoid homomorphism.

    This implies $f(0) = 1$ and $f(k·e) = f(e)^k$.

    Therefore, $f \equiv h_{f(e)}$.

    \begin{proposition}
      For every monoid $M \in \text{\bf Mon}$:

      $$Hom_{\text{\bf Mon}}(\mathbb{N}, M) \cong \vert M \vert$$
    \end{proposition}

    \begin{proof}

      Define $i_1$ and $i_2$ as arrows in {\bf Set} as follows:
      
      \begin{eqnarray*}
        i_1  : &   Hom_{\text{\bf Mon}}(\mathbb{N}, M) &→        \vert M \vert \\      
             & f                              &\mapsto   f(e)          \\
      \end{eqnarray*}

      \begin{eqnarray*}
        i_2 : &  \vert M \vert & →      Hom_{\text{\bf Mon}}(\mathbb{N}, M) \\
            &        m        & \mapsto  h_m                          \\
      \end{eqnarray*}

      $$i_2(i_1(f)) = i_2(f(e)) = h_{f(e)} = f$$ 

      $$i_1(i_2(m)) = i_1(h_m) = h_m(e) = m$$

      $i_1$ is an isomorphism in {\bf Set} with inverse $i_2$.

      Therefore, $Hom_{\text{\bf Mon}}(\mathbb{N}, M)$ and $\vert M \vert$ are
      isomorphic.
      
    \end{proof}

    \begin{corollary}
      The forgetful functor $U : \text{\bf Mon} → \text{\bf Set}$
      is representable.

    \end{corollary}
    \begin{proof}
      
      $$U(–) \equiv \vert – \vert \cong Hom_{\text{\bf Mon}}(\mathbb{N}, –)$$
    \end{proof}

    \begin{corollary}
      $U$ preserves all small products.
    \end{corollary}

    \begin{proof}

      This follows directly from Proposition 2.20.

      \begin{align*}
       U(A \times B) &= Hom(\mathbb{N}, A \times B) \\
                     &= Hom(\mathbb{N}, A) \times Hom(\mathbb{N}, B) \\
                     &= U(A) \times U(B) \\
      \end{align*}

    \end{proof}

      Non-small products can't be preserved because $U(A \times B)$
      is not well defined in that case.

\end{enumerate}


\end{document}
  
