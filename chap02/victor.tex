\documentclass{article}

\usepackage{xltxtra}
\usepackage{amsmath}
\usepackage{tikz}
\usepackage{unicode-math}
\usepackage{fontspec}
\setmathfont{xits-math.otf}

\tikzset{node distance=2cm, auto}

\author{Víctor López Juan}
\title{Chapter 1}

\begin{document}

\begin{enumerate}
  \item[12.]
    \begin{lemma}\label{lem:one}
      For every filter $F$, $1 ∈ F$.
    \end{lemma}
    \begin{proof}
      F is non empty. Let $a ∈ F$. By defintion of Boolean Algebra,
      $1 ∈ B$, $a ≤ 1$. By definition of filter, $1 ∈ F$.
    \end{proof}
    
    \begin{lemma}
      The intersection of two filters F₁, F₂ is a filter F.
    \end{lemma}
    \begin{proof}
      By lemma \ref{lem:one}, 1 ∈ F₁ ∩ F₂ = F, so F is non-empty.

      Let $a ∈ F$, $b ∈ B$, $a ≤ b$.

      By the definition of filter, $a ∈ F₁ ⇒ b ∈ F₁$, and $a ∈ F₂ ⇒ b ∈ F₂$.
      Therefore, $b ∈ F$.

      Analogously, $a ∈ F$ and $b ∈ F$ implies $a ∧ b ∈ F$.
    \end{proof}

    \begin{lemma}
      A boolean algebra can be characterized by the following properties:
    \end{lemma}
    
    
    
    \begin{lemma}
     Let F be a filtter
     F is an ultrafilter iff for every b ∈ B, b ∈ F <=> ¬b !∈ F
    \end{lemma}

    \begin{proof}
      \begin{description}
        \item[$⇒$]

          Let F be an ultrafilter.

          Let ¬F = {¬b | b ∈ F}. F and ¬F are disjoint.

          Let R = B\(F ∪ ¬F).

          If R = ∅, then, we're finished.

          Else, let x ∈ R. x !∈ F; therefore, for all y ∈ F, ¬(y ≤ x),
          which implies x ≤ y.

          /Observation:/ For every filter, 1 ∈ F. Therefore, 0 !∈ R, so
          x ≠ 0.

          Let $F' = { c | x ≤ c }$

          \begin{enumerate}
            \item F' is a filter.
            \item 0 !∈ F', therefore, F' is proper.
            \item F ⊂ F'
          \end{enumerate}

          Therefore, F is not an ultrafilter. Contradiction.
          
       \item[$⇐$]

         $b ≠ ¬b$ for every b ∈ B.

         Assume $b = ¬b$. Then, b ∧ b = 0, therefore, b = 0.

         0 = ¬0.

         Now, for all a, $a ∧ 0 ≤ a ∧ 0$. Therefore, $a ∧ 0 ≤ 0$ (by definition
         of boolean algebra), which implies $a ∧ 0 = 0$ (by antisymmetry).

         But this implies $(a ∧ ¬0) = 0$, and, therefore, a ≤ 0.

         So there can't be

         Now, if F satisfies the property, and F is a filter, then, it's
         maximal: any bigger filter F' must contain some β !∈ F, so
         ¬β ∈ F. That implies 0 ∈ F'. If 0 ∈ F', all elements greater
         than 0 are in F'; which means all the elements in the algebra.

      \end{description}
         
    \end{proof}

    \begin{lemma}
      There is an isomorphism between homomorphisms $h : B → 2$ and
      ultrafilters.
    \end{lemma}

    \begin{proof}
      Let M be the canonical isomorphism between subsets of B and
      characteristic functions $B → 2$.
      
      M :     P(B)  →  (B → 2)
              U    |→   χ_U

      M^{-1} :  (B → 2) → P(B)
               f       |-> { b | b ∈ B, f(b) = 1 }


      M is biyective, and M^{-1} is it's inverse.

      We only need to prove that M maps ultrafilters to homomorfisms,
      and viceversa for M^{-1}:
      
      \begin{itemize}
        \item

          Let U be an ultrafilter.

          χ_U(0) = 0

          χ_U(1) = 1

          Complement:
          
          χ_U(¬a) = 0 iff χ_U(a) = 1 , therefore : ¬χ_U(a) = χ_U(¬a)

          Meet:
          
          if a !∈ U or b !∈ U, then a ∧ b !∈ U (because U is a filter,
          and $a ∧ b ≤ a$, $a ∧ b ≤ b$).

          If they both belong, then their intersection belongs, because
          U is a filter.

          Therefore, χ_U(a ∧ b) = χ_U(a) ∧ χ_U(b).

          For join:

          If a or b ∈ U, then a ≤ a ∨ b or b ≤ a ∨ b, so a ∨ b ∈ U.

          Assume that a ∨ b ∈ U. If a ∈ U or b ∈ U, then we're finished.

          Else, ¬a ∈ U, and ¬b ∈ U (because U is ultrafilter).

          Also, (¬a ∧ ¬b) ∈ U, because U is filter.

          
          % TODO, prove distribuitivity, prove that (¬a ∧ ¬b) ≤ ¬(a ∨ b), done (because ultrafilter, and (a ∨ b) ∈ u ).
          
          + For order:

          Assume a ≤ b. Then, either:

          a ∈ U. Then b ∈ U, χ_U(a) = 1 ≤ 1 = χ_U(b)
          
          a !∈ U. Then, χ_U(a) = 0 ≤ χ_U(b)
          
        \itemize

          Let h be an homomorfism. Then, let F = h^{-1}(1):

          \begin{itemize}
            \item For every b in B, if h(b) = 0, then h(¬b) = ¬h(b) = 1.
              Therefore, either b in F, or ¬b in F.

            \item F is a filter:

              if a ∈ F, a ≤ b, then h(a) = 1 ≤ h(b) ⇒ b ∈ F
              if a, b ∈ F, h(a) = h(b) = 1, so 1 = h(a) ∧ h(b) = h(a ∧ b) = 1.
              
          \end{itemize}

          Therefore, F is an ultrafilter.

      \end{itemize}


\end{enumerate}


\end{document}
  
