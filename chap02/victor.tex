% For plaintext use only
\documentclass{article}

\usepackage{xltxtra}
\usepackage{amsmath}
\usepackage{amssymb}
\usepackage{amsthm}
\usepackage{tikz}
\usepackage{unicode-math}
\usepackage{fontspec}
\setmathfont{xits-math.otf}

\newtheorem{proposition}{Proposition}
\newtheorem{corollary}{Corollary}

\tikzset{node distance=2cm, auto}

\author{Víctor López Juan}
\title{Chapter 2}

\begin{document}

\begin{enumerate}
  \item[12.]

    A boolean algebra is a tuple $B = (\vert B \vert, ≤, ∧, ∨, ¬, BOT, TOP)$ which
    fulfils the following conditions for every $a$, $b$, $c$ $ ∈ B$.
    \begin{itemize}
      \item $(B, ≤)$ is a poset.
      \item $BOT ≤ a ≤ TOP$
      \item $a ≤ c$ and $b ≤ c$ iff $a ∨ b ≤ c$
      \item $c ≤ a$ and $c ≤ b$ iff $c ≤ a ∧ b$
      \item $a ≤ ¬b$ iff $a ∧ b = BOT$
      \item $¬¬a = a$
    \end{itemize}

    Boolean algebras together with their homomorphisms form a category.
    The boolean algebra $\text{\bf 2}$ is the initial algebra, where
    $\vert \text{\bf 2} \vert = \{BOT, TOP\}$.


    For a boolean algebra $B$, a filter $F \subsetof B$ is a set such
    that, for every $a$, $b$:

    \begin{itemize}
      \item $a \in F$ and $a ≤ b$ ⇒ $b ∈ F$
      \item $∧(F,F) \subsetof F$
    \end{itemize}

    $B \subsetof B$ is the trivial filter. 
    The set of non-trivial filters can be ordered by inclusion; a maximal element
    is an ultrafilter.

   \begin{enumerate}
     \item Every homomorphism $h : B → \text{\bf 2}$ corresponds to an
       ultrafilter in B.

       Let $U = h^{-1}(TOP)$.

       First, prove that $U$ is a filter.

       \begin{itemize}
         \item
           Let $a ∈ U$. Then $h(a) = TOP$

           Let $b ∈ B$ such that $a ≤ b$. $h$ is an homomorphism, which
           preserves the ordering therefore $TOP = h(a) ≤ h(b)$.

           On the other hand, $h(b) ≤ TOP$ because $\text{\bf 2}$ is a
           boolean algebra. So, $h(b) = TOP$, and $b ∈ U$.

         \item

           Let $a, b ∈ u$. Then $h(a), h(b) = TOP$.

           $h$ is an homomorphism; which implies $h(a ∧ b) = h(a) ∧ h(b) = TOP ∧ TOP$.

           For all elements x, $x ≤ TOP$. Therefore, we have

           $$TOP ≤ TOP ∧ TOP ≤ TOP$$

           Finally, $h(a ∧ b) = TOP$, so $(a ∧ b) ∈ U$.

      \end{itemize}

      $U$ is non-trivial, as $h(BOT) = BOT ≠ TOP$.

      Now, assume that $U$ is not an ultrafilter. Then, there exists
      some filter $U^\prime$, $U^\prime \subsetneq U$, $U^\prime ≠ B$.
           
      Let $x ∈ U^\prime$. Then, $h(x) = BOT$, and
      $h(¬x) = TOP$, so $¬x ∈ U \subset U^\prime$.

      Because $U^\prime$ is a filter, and $x ≤ x$, $x ∧ ¬x = BOT$, and
      $BOT ∈ U^\prime$. But, in this case, for every $x ∈ B$, $BOT ≤ x$
      and $x ∈ U^\prime$. Therefore, $U^\prime = B$, which is a contradiction.

   \item Every ultrafilter $U$ in $B$ corresponds to an homomorphism $h : B → \text{\bf 2}$.
     
     Let $h(x) = TOP$ iff $x ∈ U$.

     
      

     %% TODO: Ultrafilter properties
  \item[18.]

    We want to represent an arbitrary monoid as a set of homomorphisms.
    
    Let $(\mathbb{N}, +, 0)$ be the free, commutative monoid generated
    by one element $e$.

    $$\mathbb{N} = \{ 0, e, 2·e, 3·e, …, k·e, … \}$$


    Let $(M,·,1)$ be a monoid. For every $m \in M$, define the
    following mapping.

    \begin{align*}
    h_m & : & \mathbb{N} & →        & M   \\
        &   &     k·e    & \mapsto  & m^k \\
    \end{align*}

    In particular,

    $$h(0) = h(0·e) = m^0 = 1$$ 
    
    $$h_m(k·e + j·e) = h_m((k+j)·e) = m^(k+j) = m^k · m^j = h_m(k·e) + h_m(j·e)$$

    Therefore, $h_m$ is a monoid homomorphism.

    Now, let $f : \mathbb{N} → M$ be an arbitrary monoid homomorphism.

    This implies $f(0) = 1$ and $f(k·e) = f(e)^k$.

    Therefore, $f \equiv h_{f(e)}$.

    \begin{proposition}
      For every monoid $M \in \text{\bf Mon}$:

      $$Hom_{\text{\bf Mon}}(\mathbb{N}, M) \cong \vert M \vert$$
    \end{proposition}

    \begin{proof}

      Define $i_1$ and $i_2$ as arrows in {\bf Set} as follows:
      
      \begin{align*}
        i_1 & : & Hom_{\text{\bf Mon}}(\mathbb{N}, M) & →        & \vert M \vert \\      
            &   & f                                \mapsto  & f(e)          \\
      \end{align*}

      \begin{align*}
        i_2 & : & \vert M \vert & →     & Hom_{\text{\bf Mon}}(\mathbb{N}, M) \\
            &   &      m        & \mapsto & h_m                          \\
      \end{align*}

      $$i_2(i_1(f)) = i_2(f(e)) = h_{f(e)} = f$$ 

      $$i_1(i_2(m)) = i_1(h_m) = h_m(e) = m$$

      $i_1$ is an isomorphism in {\bf Set} with inverse $i_2$.

      Therefore, $Hom_{\text{\bf Mon}}(\mathbb{N}, M)$ and $\vert M \vert$ are
      isomorphic.
      
    \end{proof}

    \begin{corollary}
      The forgetful functor $U : \text{\bf Mon} → \text{\bf Set}$
      is representable.

    \end{corollary}
    \begin{proof}
      
      $$U(–) \cong Hom_{\text{\bf Mon}}(\mathbb{N}, –)$$
    \end{proof}

    \begin{corollary}
      $U$ preserves all small products.
    \end{corollary}

    \begin{proof}

      This follows directly from Proposition 2.20.

      \begin{align*}
       & U(A \times B) \\
      =& Hom(\mathbb{N}, A \times B) \\
      =& Hom(\mathbb{N}, A) \times Hom(\mathbb{N}, B) \\
      =& U(A) \times U(B)
      \end{align*}

    \end{proof}

      Non-small products can't be preserved because $U(A \times B)$
      is not well defined.

      
    

    
    


\end{enumerate}


\end{document}
  
