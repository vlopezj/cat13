% chapter=4
% exercise=8
% author=Fabian

We verify the congruence axioms. Let $f \sim g$ for two arrows $f : A → B, g : A' → B'$ in $\cat{D}$, then:

\begin{enumerate}
	\item $\ \sim\ $ is obviously reflexive, symmetric and transitive since it is defined solely in terms of equality between domains, codomains and map images, which are equivalence relations and, therefore, reflexive, symmetric and transitive.
	\item By definition of $\ \sim\ $ we have $A = dom(f) = dom(g) = A'$ and $B = cod(f) = cod(g) = B'$.
	\item Let $a : B → X, b : Y → A$ arbitrary arrows in $\cat{D}$ for any objects $X, Y$ and show $a ∘ f ∘ b \sim a ∘ g ∘ b$. Firstly, observe that the compositions exist and have the same domain and codomain. Secondly, observe $H(a ∘ f ∘ b) = H(a) ∘ H(f) ∘ H(b) = H(a) ∘ H(g) ∘ H(b) = H(a ∘ g ∘ b)$ for any category $\cat{E}$ and functor $H : \cat{D} → \cat{E}$ such that $H ∘ F = H ∘ G$. The first and last equality hold because functors respect arrow composition. Lastly, the second equality holds because $f \sim g$ implies $H(f) = H(g)$ for such $H$ by definition. Finally, $a ∘ f ∘ b \sim a ∘ g ∘ b$ because the $\sim$-axioms are satisfied.
\end{enumerate}

For $\qcat{D}{\sim}$ to be the coequalizer of $F$ and $G$ in $\cat{Cat}$, firstly, we recall the quotient functor $π : \cat{D} → \qcat{D}{\sim}$ and show $π ∘ F = π ∘ G$. Since $π$ is the identity mapping on objects and $F$ and $G$ agree on objects by assumption, we need only to show $π(F(a)) = π(G(a))$ for an arbitrary arrow $a$ in $\cat{C}$. The latter is the case if and only if $F(a) \sim G(a)$, which is the case if and only if $H(F(a)) = H(G(a))$ for any functor $H$ as above. Since $H ∘ F = H ∘ G$ is assumed already, in particular $H(F(a)) = H(G(a))$ holds and, hence, $F(a) \sim G(a)$. Secondly, we show that $\qcat{D}{\sim}$ satisfies the universal property of a coequalizer and assume $\tuple{\cat{Z}}{Z}$ to be another object and functor pair that makes $F$ and $G$ equal. Define the functor $U : \qcat{D}{\sim} → \cat{Z}$ as follows:

\begin{itemize}
	\item $U(o) := Z(o)$ for $o ∈ (\qcat{D}{\sim})_1 = \cat{D}_1$
	\item $U([a]) := Z(a)$ for $a ∈ \cat{D}_0$
\end{itemize}

This is well-defined because $[a] = [b]$ ($⇔ a \sim b$) and $Z ∘ F = Z ∘ G$ imply $Z(a) = Z(b)$. Furthermore, the functor axioms can be verified easily (we have $[a ∘ b] = [a] ∘ [b]$). Lastly, $U$ lets commute the diagram below, which follows immediately from the definition of $U$ above.

\begin{tikzpicture}
  \node (C) {$C$};
  \node (D) [right of=C] {$\cat{D}$};
  \node (D/) [right of=D] {$\qcat{D}{\sim}$};
  \node (Z) [below of=D/] {$\cat{Z}$};
  \draw[->, bend left] (C) to node {$F$} (D);
  \draw[->, bend right] (C) to node {$G$} (D);
  \draw[->] (D) to node {$π$} (D/);
  \draw[->, dashed] (D/) to node {$U$} (Z);
  \draw[->] (D) to node {$Z$} (Z);
\end{tikzpicture}

Any other such $U' : \qcat{D}{\sim} → \cat{Z}$ that lets the diagram commute agrees with our definition, which makes $U$ unique: $U'(o) = U(π(o)) = Z(o) (= U(o))$ and $U'([a]) = U'(π(a)) = Z(a) (= U([a]))$ for $o, a ∈ \cat{D}$. \hfill $\Box$
