% For plaintext use only
\documentclass{article}

\usepackage{xltxtra}
\usepackage{amsmath}
\usepackage{tikz}
\usepackage{unicode-math}
\usepackage{fontspec}
\setmathfont{xits-math.otf}

\tikzset{node distance=2cm, auto}

\author{Víctor López Juan}
\title{Chapter 4}

\begin{document}

\begin{enumerate}
  \item[4.]
    % Every monoid in the category of groups is an internal group.

    % Internal group:

    % In the same way that groups in the category of groups are abelian
    % internal groups, we will have that monoids in the category of groups
    % are internal groups (not necesarily abelian).

    \begin{tikzpicture}
      \node $G \cross G$ GG
      \node $G$          G \rightof GG
      \node $1$          T \rightof G
      \arrow GG -> G \label \star
      \arrow 1  -> G \label u
    \end{tikzpicture}

    Observe that $u$ is the inclusion of the trivial group in $G$.

    Now:

    \begin{itemize}
      \item The underlying set of the monoid is $G$. The operation is $m$.
      \item $m$ is associative, because $m$ is part of a monoid in the
        category.

      \item $m$ respects identity, again because it's a monoid.

      \item $m$ is a group homomorphism.

        $m(g₁·h₁, g₂·h₂) = m((g₁,g₂)·(h₁,h₂)) = m(g₁,g₂)·m(h₁,h₂)$
                                              

        Therefore, the operation $m$ is the same as the original operation
        in the group, and is therefore, an inverse. Also, only commutative
        groups can form a monoid in Group.
        

  \item[8.]

    \begin{itemize}
      \item It is a congruence:

        % f ~ g implies dom,cod(f) = dom,cod(g) %
        % f ~ g implies bfa ~ bga

        \item By definition, the domain and the codomain match.
        \item
          Let f, g such that $f ~ g$.
          

          \begin{itemize}
            \item $~$ is an equivalence relation. This is derived from the
              fact that equality is an equivalence relation, and the double
              implication.

            \item $dom,cod(bfa) = \lbracket dom(a), cod(b) \rbracket = dom,cod(bga)$$
            \item
              
              Let $H : D → E$ such that $HF = HG$. Then $H(f) = H(g)$

              H(bfa) = H(b)H(f)H(a) = H(b)H(g)H(a) = H(bga)
              
          \end{itemize}

        \item

          % D/~ is the coequalizer of F and G

          \begin{tikzpicture}
            \node $C$ C
            \node $D$ D
            \node $\nicefrac{D}{\tilde}$  DQ

            \arrow C -> D $F$
            \arrow C -> D $G$
            \arrow D -> DQ $P$
            
          \end{tikzpicture}

          \begin{itemize}
            \item % PF = PG

              P(F(C)) = P(G(C)), because F(C) = G(C)


              \begin{itemize}
                dom,cod(Fα) = dom,cod(Gα)

                If HF = HG, then H(Fα) = H(Gα).

              \end{itemize}
          
              Therefore: F(α) ~ G(α)

              Therefore: PF(α) = PG(α)

              Now, if there is a functor $Z : D -> X$, such that
              $ZF = ZG$, then there is a unique functor $\bar{Z} :
              \nicefrac{D}{~} → X$, such that

              $\bar{Z} \comp P = Z$. 

              $\bar{Z} = Z \comp I$, where $I$ is a choice function
              from $\nicefrac{D}{~}$ to the original set.

              \begin{itemize}
                \item
                    For every $x$, let $\bar{x} = I(P(x))$
                    
                    \begin{equation}
                    \bar{Z} P(x) = Z (I P(x)) =          % projection
                                    
                                 = Z (I P(\bar{x}) )) =  % x ~ \bar{x}

                                 = Z (\bar{x})           % \bar{x} is the representative

                                 = Z (x)                 % x ~ \bar{x}
                    \end{equation}  
                    
                \item
                    Uniqueness:
                    
                    Let $\bar{Z}^\prime : \nicefrac{D}{~} → X$ such
                    that $\bar{Z}^\prime \comp P = Z$.

                    
                    Then $\bar{Z}^\prime(x) = \bar{Z}^\prime(P(I(x))) =
                                          = Z(I(x)) = \bar{Z}(x)$.
             \end{itemize}
        \end{itemize}
          
\end{enumerate}

\end{document}
  
