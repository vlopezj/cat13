%use lualatex/xelatex to compile this
\documentclass{scrartcl}
\usepackage{polyglossia}
\setdefaultlanguage{english}

\usepackage{marvosym}% für den Smiley :-)
\def\contradiction{\quad\text{\Large\Lightning}}
\usepackage{graphicx}
\usepackage{float}
\usepackage{listings}
\usepackage{color}
\usepackage{enumerate}

\lstset{%
  literate={ö}{{\"o}}1%
           {ä}{{\"a}}1%
           {ü}{{\"u}}1%
           {Ö}{{\"O}}1%
           {Ä}{{\"A}}1%
           {Ü}{{\"U}}1%
           {ß}{{\ss}}1%
}

\definecolor{bggray}{rgb}{0.90,0.90,0.90}
\definecolor{dkgreen}{rgb}{0.00,0.50,0.00}
\definecolor{mauve}{rgb}{0.50,0.00,0.30}
\definecolor{darkGray}{rgb}{0.40,0.40,0.40}
\lstset{%
  language=C++,                   % the language of the code
  basicstyle=\small\ttfamily,
  numbers=none,                   % where to put the line-numbers
  backgroundcolor=\color{bggray}, % choose the background color. You must add \usepackage{color}
  showspaces=false,               % show spaces adding particular underscores
  showstringspaces=false,         % underline spaces within strings
  showtabs=false,                 % show tabs within strings adding particular underscores
  frame=single,                   % adds a frame around the code
  tabsize=4,                      % sets default tabsize to 2 spaces
  breaklines=true,                % sets automatic line breaking
  breakatwhitespace=true,         % sets if automatic breaks should only happen at whitespace
  title=\lstname,                 % show the filename of files included with \lstinputlisting;
                                  % also try caption instead of title
  keywordstyle=\color{blue},      % keyword style
  commentstyle=\color{dkgreen},   % comment style
  stringstyle=\color{mauve},      % string literal style
  escapeinside={\%*}{*)},         % if you want to add a comment within your code
}

\usepackage{amsmath,amsfonts,amsthm,amssymb,bbm,dsfont,mathrsfs}
\delimitershortfall=-3pt %makes parentheses grow larger automagically

\DeclareMathOperator*{\argmin}{argmin}
\DeclareMathOperator{\argmax}{argmax}
\DeclareMathOperator*{\cupdot}{\stackrel{{}_\bullet}{\cup}}
\DeclareMathOperator*{\bigcupdot}{\stackrel{{}_\bullet}{\bigcup}}
\def\LHS{&\phantom{{}=}} % First line of align environment with no relation symbol should still be align with the other lines. This is a dirty fix.
\usepackage{txfonts}

%Komma und Punktabstände anpassen:
%\mathcode`,="013B %nervt, außerdem braucht niemand Kommazahlen.
%\mathcode`.="613A

\author{Stefan Walzer}

%MENGEN Symbole
\newcommand{\R}{\ensuremath{\mathbb R}}
\newcommand{\N}{\ensuremath{\mathbb N}}
\newcommand{\Z}{\ensuremath{\mathbb Z}}
\newcommand{\Q}{\ensuremath{\mathbb Q}}
\newcommand{\C}{\ensuremath{\mathbb C}}
\newcommand{\K}{\ensuremath{\mathbb K}}

%Quantoren
\newcommand{\E}{\ensuremath{\exists}}
\newcommand{\A}{\ensuremath{\forall}}

%Pfeile
\newcommand{\LA}{\ensuremath{\Leftarrow}}
\newcommand{\RA}{\ensuremath{\Rightarrow}}
\newcommand{\la}{\ensuremath{\leftarrow}}
\newcommand{\ra}{\ensuremath{\rightarrow}}
\newcommand{\LRA}{\ensuremath{\Leftrightarrow}}
\newcommand{\lra}{\ensuremath{\leftrightarrow}}
\newcommand{\LLRA}{\ensuremath{\Longleftrightarrow}}
\newcommand{\llra}{\ensuremath{\longleftrightarrow}}

%spitze Klammern
\newcommand{\lan}{\ensuremath{\langle}}
\newcommand{\ran}{\ensuremath{\rangle}}


%Symbole
\newcommand{\veps}{\ensuremath{\varepsilon}}

%Verknüpfungen
\newcommand{\AND}{\wedge} %und
\newcommand{\OR}{\vee} %oder
\newcommand{\UN}{\cup} %union
\newcommand{\IS}{\cap} %intersection

%sonstiges
\def\abs#1{\left|#1\right|}
\def\norm#1{\lVert#1\rVert}
\def\floor#1{\left\lfloor#1\right\rfloor}
\def\ceil#1{\left\lceil#1\right\rceil}
\def\enquote#1{\glqq #1 \grqq}

\def\powset#1{\mathscr{P}\left(#1\right)}
\def\set#1{\left\{#1\right\}}
\def\setgen#1#2{\left\{#1\;\middle|\;#2\right\}}

%vector
\newcommand{\vect}[1]{\begin{pmatrix}#1\end{pmatrix}}
\newcommand{\svect}[1]{\scalebox{0.7}{$\begin{pmatrix}#1\end{pmatrix}$}}

%Fonts:

\setmainfont[Ligatures=TeX,SmallCapsFont={Latin Modern Roman Caps}]{Georgia}
\setsansfont{Segoe UI}
\setmonofont{Consolas}
\usepackage{unicode-math}
\setmathfont{xits-math.otf}

\usepackage{newunicodechar}
\newunicodechar{Ø}{\emptyset}

\begin{document}
    %\section*{Exercise 1}
    %
    %We show that every congruence relation corresponds to a normal group and vice versa.
    %
    %\begin{desription}
    %    \item[$⊂$] Let $\sim$ be a congruence relation in a group $G$. Define $N \coloneqq \setgen{g ∈ G}{g \sim e}$ where $e$ is the neutral element of $G$.
    %    
    %    $N$ is certainly a normal subgroup of $G$ since if $n ∈ N$ then:
    %    \begin{description}
    %        \item[$N$ contains inverses] $n \sim e ⇒ e · n · n^{-1} \sim e · e · n^{-1} ⇒ e \sim n^{-1}$
    %        \item[$N$ is closed under the group operation] $n_1 \sim e ⇒ e · n_1 · n_2 \sim e · e · n_2 ⇒ n_1 · n_2 \sim n_2 \sim e$.
    %        \item[$N$ is closed under conjugation] $n \sim e ⇒ g · n · g^{-1} \sim g · e · g^{-1} = e$.
    %    \end{description}
    %    \item[$⊃$] 
    %\end{desription}
    
    \section*{Excercise 3} Let $(G,+)$ be an abelian group. We have to show, that $+ : G × G → G$ and $·^{-1} : G → G$ are group homomorphisms.
    
    This means that:
    
    \[ (g_1 + h_1) + (g_2 + h_2) = (g_1 + g_2) + (h_1 + h_2) \]
    \[ (g_1 + g_2)^{-1} = g_1^{-1} + g_2^{-1} \]
    
    The former is clear since $+$ is commutative, the latter is also clear since $(g_1 + g_2)$ is the inverse to both sides (again using commutativity).
    
    The remaining properties of $+$, $i$ and $u$ are simple to check since they are already implicit in the fact that $(G,+)$ is a group.
    
    \section*{Excercise 4} Let $M$ be a monoid in the category of groups, i.e. let $(M, ·)$ be a group $\star: M×M → M$ an associative group homomorphism and $e$ the neutral (generalised) element of $M$. Since $\star$ is a group homomorphism we have (mixing infix and prefix notation when it makes things look easier):
    
    \[ (g_1· h_1) \star (g_2 · h_2) = \star(g_1· h_1, g_2 · h_2) = \star((g_1,g_2) · (h_1, h_2)) = \star(g_1,g_2) · \star(h_1,h_2) = (g_1 \star g_2) · (h_1 \star h_2) \]
    
    By Proposition 4.5 (Eckmann-Hilton) we now know that $\star = ·$ and that this operation is commutative.
    
    Therefore $M$ was an abelian group and the operation $·^{-1}$ is a group homomorphism. $M$ can therefore be represented as an internal group (since exactly abelian groups are internal groups in {\bf Grp}).
    
    \section*{Excercise 7}
    
    This is simple:\\
    
    $f \sim f'$ implies $g \circ f \circ id \sim  g \circ f' \circ id$.
    
    $g \sim g'$ implies $id \circ g \circ f' \sim  id \circ g' \circ f'$.
    
    Using transitivity we get $g \circ f \sim g' \circ f'$.\\
    
    \noindent This proofs the claim.
\end{document}