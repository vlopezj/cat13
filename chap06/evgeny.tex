\documentclass[12pt]{article}

\usepackage{amsmath}
\usepackage{amsthm}
\usepackage{amssymb}
\usepackage{fancyhdr}
\usepackage{tikz}
\usepackage{float}
\pagestyle{plain}
\theoremstyle{definition}

\tikzset{node distance=3cm, auto}

\author{Evgeny Kotelnikov}
\title{Exercises for Chapter 6}
\date{}

\begin{document}
\maketitle

\begin{enumerate}
  \item[3.]
    Straightforwardly from the definition of an exponential:
    \begin{enumerate}
      \item $\tilde{e} : B^A \to B^A = 1_{B^A}$,
      \item $\tilde{1} : A \to (A \times B)^B = a \mapsto \lambda b : B . \langle a , b \rangle$,
      \item $\tilde{\epsilon \circ \tau} : A \to B^{B^A} = a \mapsto \lambda f : B^A . f(a)$.
    \end{enumerate}

  \item[9.]
    \newtheorem*{lemma}{Lemma}
    \begin{lemma}
      \vspace{-1.9em}
      $$1 \times A \cong A.$$
    \end{lemma}
    \begin{proof}
      Since $1 \times A$ is a product, there exists unique $u : A \to 1 \times A$ such that
      \begin{equation}
        \label{eq:f=pu}
        f = \pi_1 \circ u
      \end{equation}
      and
      \begin{equation*}
        1_A = \pi_2 \circ u.
      \end{equation*}
      
      \begin{figure}[H]
        \centering
        \begin{tikzpicture}
          \node (1)               {$1$};
          \node (1*A) [right of=1] {$1 \times A$};
          \node (A)   [right of=1*A] {$A$};
          \node (A')  [below of=1*A] {$A$};

          \draw[<-] (1) to node {$\pi_1$} (1*A);
          \draw[->] (1*A) to node {$\pi_2$} (A);
          \draw[dashed,->] (A')  to node {$f$}     (1);
          \draw[<-] (A)   to node {$1_A$}   (A');
          \draw[dashed,->] (A') to node {$u$} (1*A);
        \end{tikzpicture}
      \end{figure}
      Composing both sides of 1 with $\pi_2$ we get $f \circ \pi_2 = \pi_1 \circ u \circ \pi_2$. At the same time, $f \circ \pi_2 = \pi_1$. Therefore, $\pi_1 = \pi_1 \circ u \circ \pi_2$ and $1_{1 \times A} = u \circ \pi_2$. We have now that $1_A = \pi_2 \circ u$ and $1_{1 \times A} = u \circ \pi_2$. From the uniqueness conditions on $u$ and $f$ (coming from the fact, that $1$ is a terminal object) we have that $u$ and $\pi_2$ are isomorphisms and $1 \times A \cong A$.
    \end{proof}

    From the lemma it immediately follows that $$\mathrm{Hom}_{\mathbf{C}}(1 \times A, B) \cong \mathrm{Hom}_{\mathbf{C}}(A, B).$$

    From the property of $1 \to B^A$ exponential (Awodey, 6.1) we have $$\mathrm{Hom}_{\mathbf{C}}(1 \times A, B) \cong \mathrm{Hom}_{\mathbf{C}}(1, B^A).$$

    By transitiviy $$\mathrm{Hom}_{\mathbf{C}}(A, B) \cong \mathrm{Hom}_{\mathbf{C}}(1, B^A),$$ that is, there is a bijection between arrows $A \to B$ and $1 \to B^A$.

\end{enumerate}

\end{document}
