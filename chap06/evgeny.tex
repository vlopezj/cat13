\documentclass[12pt]{article}

\usepackage{amsmath}
\usepackage{amsthm}
\usepackage{amssymb}
\usepackage{fancyhdr}
\usepackage{tikz}
\usepackage{float}
\pagestyle{plain}
\theoremstyle{definition}

\tikzset{node distance=3cm, auto}

\author{Evgeny Kotelnikov}
\title{Exercises for Chapter 6}
\date{}

\begin{document}
\maketitle

\begin{enumerate}
  \item[3.]
    Straightforwardly from the definition of an exponential:
    \begin{enumerate}
      \item $\tilde{e} : B^A \to B^A = 1_{B^A}$,
      \item $\tilde{1} : A \to (A \times B)^B$,
      \item $\tilde{\epsilon \circ \tau} : A \to B^{B^A}$.
    \end{enumerate}

  \item[9.]
    In $\lambda$-calculus terms we are basically asked to show that for every types $A$ and $B$ there exists a pair function between types $\mathbf{Unit} \to A \to B $ and $A \to B$. Such pair of functions is $\lambda f : \mathbf{Unit} \to A \to B. f(1)$ and $\lambda f : A \to B. \lambda u : \mathbf{Unit}. f$ with $1$ being inhabitant of $\mathbf{Unit}$ type (corresponding to terminal object of the category).

\end{enumerate}

\end{document}
