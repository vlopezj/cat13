% For plaintext use only
\documentclass{article}

\usepackage{xltxtra}
\usepackage{amsmath}
\usepackage{tikz}
\usepackage{unicode-math}
\usepackage{fontspec}
\setmathfont{xits-math.otf}

\tikzset{node distance=2cm, auto}

\author{Víctor López Juan}
\title{Chapter 2}

\begin{document}

\begin{enumerate}
  \item[9.]
    $\bar{h} = h \comp i$ is a monoid homomorfism.
    \begin{itemize}
      \item
        $ \bar{h}(1_M) = h(i(1_M)) = h(\rbracket 1_M \lbracket)$
        % empty product is unit
        $= h(μ(\lbracket \lbracket \rbracket \rbracket)) = $
        $= h(ε(\lbracket \lbracket \rbracket \rbracket)) = $
        $= h(\lbracket \rbracket) = $
        % h is homomorfism
        $= 1_N $
      \item

        h is a monoid homomorfism

        $\bar{h}(xy) = h(i(x·y)) = h(\lbracket x·y \rbracket) =
         h(μ(\lbracket \lbrackey x, y \rbracket \rbracket)) =
         h(ε(\lbracket \lbrackey x, y \rbracket \rbracket)) =
         h(\lbrackey x, y \rbracket)) =
         h(\lbracket x \rbracket \cdot \lbracket y \rbracket) =
         % h is homomorfism
         h(\lbracket x \rbracket) \cdot h(\lbracket y \rbracket) =
         % definition of iu¡
         h(i(x)) \cdot h(i(y)) =
         % definition of hbar
         \bar{h}(x) \cdot \bar{h}(y) =

  \item[10.]

        Let $(e,M)$ be the coequalizer of $μ$ and $ε$, and $F$ be the
        forgetful functor $Mon → Set$.

        $F(\pi)\comp F(μ) = F(\pi \comp μ) = F(\pi \comp ε) = F(\pi) \comp F(ε)$

        Also, it is universal with respect to this property.

        Let $i$ be the inclusion of generators into TM $i : M → TM$,
        $i(x) = \lbracket x \rbracket$.
        
        Let there be a set $X$, and an arrow $z : TM → X$. Then there's
        an arrow $\bar{z} : M → X$, $\bar{z} = z \comp i$.

        Is this arrow unique in set?

        Assume $\bar{z}^\prime$ such that $\bar{z}^\prime \comp F(\pi) = z$.

        Then:

        $\bar{z}^\prime(x) = \bar{z}^\prime(F(\pi)(\lbracket x \rbracket)) =$
        $                 = z(\lbracket x \rbracket)                  =$
        $                 = \bar{z}(F(\pi)(\lbracket x \rbracket))       =$
        $                 = \bar{z}(x)$.

        Therefore, $\bar{z}^\prime = \bar{z}$

\end{document}
  
