% For plaintext use only
\documentclass{article}

\usepackage{xltxtra}
\usepackage{amsmath}
\usepackage{amssymb}
\usepackage{amsthm}
\usepackage{tikz}
\usepackage{unicode-math}
\usepackage{fontspec}
\setmathfont{xits-math.otf}

\tikzset{node distance=2cm, auto}

\author{Víctor López Juan}
\title{Chapter 3}

\begin{document}

\begin{enumerate}
  \item[9.]
    $\bar{h} = h \circ i$ is a monoid homomorfism.
    
    \begin{proof}
    \begin{itemize}
      Monoid homomorphisms must map identity elements to each other,
      and preserve the operation.
      
      \item

        $\bar{h}$ respects the unit element.
        
        \begin{align*}
          & \bar{h}(1_M)  \\
        = & h(i(1_M)) \\
        = & h(\langle 1_M \rangle)   \\
        = & h(μ(\langle \langle \rangle \rangle))           && \text{empty product is unit} \\
        = & h(ε(\langle \langle \rangle \rangle))   \\
        = & h(\langle \rangle)                             \\
        = & 1_N                                              && \text{$h$ is an homomorfism} \\
        \end{align*}

      \item

        $\bar{h}$ respects the monoid operation.

        \begin{align*}
           \bar{h}(xy) \\
         = h(i(x·y)) = h(\langle x·y \rangle)  \\
         = & h(μ(\langle \langle x, y \rangle \rangle))  \\
         = & h(ε(\langle \langle x, y \rangle \rangle))  \\
         = & h(\langle x, y \rangle))  \\
         = & h(\langle x \rangle \cdot \langle y \rangle)  \\
         = & h(\langle x \rangle) \cdot h(\langle y \rangle)  && \text{$h$ is an homomorfism} \\
         = & h(i(x)) \cdot h(i(y))                                    && \text{definition of $i$} \\
         = & \bar{h}(x) \cdot \bar{h}(y)                              && \text{definition of $\bar{h}$} \\
         \end{align*}
     \end{itemize}
    \end{proof}
  \item[10.]

        Let $(e,M)$ be the coequalizer of $μ$ and $ε$, and $F$ be the
        forgetful functor $Mon → Set$.

        $$F(\pi)\circ F(μ) = F(\pi \circ μ) = F(\pi \circ ε) = F(\pi) \circ F(ε)$$

        Also, it is universal with respect to this property.

        Let $i$ be the inclusion of generators into TM $i : M → TM$,
        $i(x) = \langle x \rangle$.
        
        Let there be a set $X$, and an arrow $z : TM → X$. Then there's
        an arrow $\bar{z} : M → X$, $\bar{z} = z \circ i$.

        Is this arrow unique in {\bf Set}?

        Assume $\bar{z}^\prime$ such that $\bar{z}^\prime \circ F(\pi) = z$.

        Then:

        \begin{align*}
          & \bar{z}^\prime(x) \\
        = & \bar{z}^\prime(F(\pi)(\langle x \rangle))  \\
        = & z(\langle x \rangle)                                  \\
        = & \bar{z}(F(\pi)(\langle x \rangle))                         \\
        = & \bar{z}(x) \\
        \end{align*}
 
        Therefore, $\bar{z}^\prime = \bar{z}$
\end{enumerate} 
\end{document}
   
 
