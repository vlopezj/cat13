\documentclass[12pt]{article}

\usepackage{amsmath}
\usepackage{amsthm}
\usepackage{amssymb}
\usepackage{fancyhdr}
\usepackage{tikz}
\usepackage{float}
\pagestyle{plain}
\theoremstyle{definition}

\tikzset{node distance=3cm, auto}

\author{Evgeny Kotelnikov}
\title{Exercises for Chapter 3}
\date{}

\begin{document}
\maketitle

\begin{enumerate}
  \item[8.]
    Recall the definition of \emph{projective object}.
    \newtheorem*{projective}{Definition}
    \begin{projective}
      An object $P$ is said to be \emph{projective} if for any \emph{epi} $e : E \twoheadrightarrow X$ and arrow $f : P \to X$ there is some arrow $\overline{f} : P \to E$ such that $e \circ \overline{f} = f$.
      \begin{figure}[H]
        \centering
        \begin{tikzpicture}
          \node (E) {$E$};
          \node (X) [below of=E] {$X$};
          \node (P) [left  of=X] {$P$};

          \draw[->]  (P) to node {$\overline{f}$} (E);
          \draw[->>] (E) to node {$e$} (X);
          \draw[<-]  (X) to node {$f$} (P);
        \end{tikzpicture}
      \end{figure}
    \end{projective}
    
    Dualizing.
    
    \newtheorem*{injective}{Definition}
    \begin{injective}
      An object $Q$ is said to be \emph{injective} if for any \emph{mono} $m : X \rightarrowtail M$ and arrow $g : X \to Q$ there is some arrow $\overline{g} : M \to Q$ such that $\overline{g} \circ m = g$.
      \begin{figure}[H]
        \centering
        \begin{tikzpicture}
          \node (M) {$M$};
          \node (X) [below of=M] {$X$};
          \node (Q) [left  of=X] {$Q$};

          \draw[<-]  (Q) to node {$\overline{g}$} (M);
          \draw[<-<] (M) to node {$m$} (X);
          \draw[->]  (X) to node {$g$} (Q);
        \end{tikzpicture}
      \end{figure}
    \end{injective}

    \newtheorem*{injective-poset}{Example of injective poset}
    \begin{injective-poset}
      A poset, that consists of a single element $a$ and a single order relation $a \leq a $. Any map into it produces the same $a$, therefore injective object's property trivially holds. 
    \end{injective-poset}

    \newtheorem*{noninjective-poset}{Example of non-injective poset}
    \begin{noninjective-poset}
      
    \end{noninjective-poset}

  \item[11.]
    In set-theoretic terms, a coequalizer of functions $f, g : A \to B$ is a set $Q$ and a function $q : B \to Q$ such that $q(f(x)) = q(g(x))$ for every $x \in A$.
    \begin{figure}[H]
      \centering
      \begin{tikzpicture}
        \node (A) {$A$};
        \node (B) [right of=A] {$B$};
        \node (Q) [right of=B] {$Q$};
        \node (Z) [below of=Q] {$Z$};
        \draw[->,transform canvas={yshift=0.5ex}]  (A) to node {$f$} (B);
        \draw[->,transform canvas={yshift=-0.5ex}] (A) to node[below] {$g$} (B);
        \draw[->] (B) to node {$q$} (Q);
        \draw[->] (B) to node[anchor=north] {$z$} (Z);
        \draw[->, dashed] (Q) to node {$u$} (Z);
      \end{tikzpicture}
    \end{figure}
    Let $R$ be equivalence relation that is generated by $f(x) = g(x)$ for all $x \in A$. Let $Q = B/R$ and $q$ be a projection map. It is easy to show, that $q \circ f = q \circ g$, $$q(f(x)) = [f(x)] = [g(x)] = q(g(x)).$$ To prove coequalizer's UMP, take an object $Z$ and arrow $z : B \to Z$ such that $z \circ f = z \circ g$. Define $u : Q \to Z$ as $u([y]) = z(y)$ for every $y \in B$. Straightforwadly, $u \circ q = z$, which concludes the proof.
\end{enumerate}

\end{document}
