%use lualatex/xelatex to compile this
\documentclass{scrartcl}
\usepackage{polyglossia}
\setdefaultlanguage{english}

\usepackage{marvosym}% für den Smiley :-)
\def\contradiction{\quad\text{\Large\Lightning}}
\usepackage{graphicx}
\usepackage{float}
\usepackage{listings}
\usepackage{color}
\usepackage{enumerate}

\lstset{%
  literate={ö}{{\"o}}1%
           {ä}{{\"a}}1%
           {ü}{{\"u}}1%
           {Ö}{{\"O}}1%
           {Ä}{{\"A}}1%
           {Ü}{{\"U}}1%
           {ß}{{\ss}}1%
}

\definecolor{bggray}{rgb}{0.90,0.90,0.90}
\definecolor{dkgreen}{rgb}{0.00,0.50,0.00}
\definecolor{mauve}{rgb}{0.50,0.00,0.30}
\definecolor{darkGray}{rgb}{0.40,0.40,0.40}
\lstset{%
  language=C++,                   % the language of the code
  basicstyle=\small\ttfamily,
  numbers=none,                   % where to put the line-numbers
  backgroundcolor=\color{bggray}, % choose the background color. You must add \usepackage{color}
  showspaces=false,               % show spaces adding particular underscores
  showstringspaces=false,         % underline spaces within strings
  showtabs=false,                 % show tabs within strings adding particular underscores
  frame=single,                   % adds a frame around the code
  tabsize=4,                      % sets default tabsize to 2 spaces
  breaklines=true,                % sets automatic line breaking
  breakatwhitespace=true,         % sets if automatic breaks should only happen at whitespace
  title=\lstname,                 % show the filename of files included with \lstinputlisting;
                                  % also try caption instead of title
  keywordstyle=\color{blue},      % keyword style
  commentstyle=\color{dkgreen},   % comment style
  stringstyle=\color{mauve},      % string literal style
  escapeinside={\%*}{*)},         % if you want to add a comment within your code
}

\usepackage{amsmath,amsfonts,amsthm,amssymb,bbm,dsfont,mathrsfs}
\delimitershortfall=-3pt %makes parentheses grow larger automagically

\DeclareMathOperator*{\argmin}{argmin}
\DeclareMathOperator{\argmax}{argmax}
\DeclareMathOperator*{\cupdot}{\stackrel{{}_\bullet}{\cup}}
\DeclareMathOperator*{\bigcupdot}{\stackrel{{}_\bullet}{\bigcup}}
\def\LHS{&\phantom{{}=}} % First line of align environment with no relation symbol should still be align with the other lines. This is a dirty fix.
\usepackage{txfonts}

%Komma und Punktabstände anpassen:
%\mathcode`,="013B %nervt, außerdem braucht niemand Kommazahlen.
%\mathcode`.="613A

\author{Stefan Walzer}

%MENGEN Symbole
\newcommand{\R}{\ensuremath{\mathbb R}}
\newcommand{\N}{\ensuremath{\mathbb N}}
\newcommand{\Z}{\ensuremath{\mathbb Z}}
\newcommand{\Q}{\ensuremath{\mathbb Q}}
\newcommand{\C}{\ensuremath{\mathbb C}}
\newcommand{\K}{\ensuremath{\mathbb K}}

%Quantoren
\newcommand{\E}{\ensuremath{\exists}}
\newcommand{\A}{\ensuremath{\forall}}

%Pfeile
\newcommand{\LA}{\ensuremath{\Leftarrow}}
\newcommand{\RA}{\ensuremath{\Rightarrow}}
\newcommand{\la}{\ensuremath{\leftarrow}}
\newcommand{\ra}{\ensuremath{\rightarrow}}
\newcommand{\LRA}{\ensuremath{\Leftrightarrow}}
\newcommand{\lra}{\ensuremath{\leftrightarrow}}
\newcommand{\LLRA}{\ensuremath{\Longleftrightarrow}}
\newcommand{\llra}{\ensuremath{\longleftrightarrow}}

%spitze Klammern
\newcommand{\lan}{\ensuremath{\langle}}
\newcommand{\ran}{\ensuremath{\rangle}}


%Symbole
\newcommand{\veps}{\ensuremath{\varepsilon}}

%Verknüpfungen
\newcommand{\AND}{\wedge} %und
\newcommand{\OR}{\vee} %oder
\newcommand{\UN}{\cup} %union
\newcommand{\IS}{\cap} %intersection

%sonstiges
\def\abs#1{\left|#1\right|}
\def\norm#1{\lVert#1\rVert}
\def\floor#1{\left\lfloor#1\right\rfloor}
\def\ceil#1{\left\lceil#1\right\rceil}
\def\enquote#1{\glqq #1 \grqq}

\def\powset#1{\mathscr{P}\left(#1\right)}
\def\set#1{\left\{#1\right\}}
\def\setgen#1#2{\left\{#1\;\middle|\;#2\right\}}

%vector
\newcommand{\vect}[1]{\begin{pmatrix}#1\end{pmatrix}}
\newcommand{\svect}[1]{\scalebox{0.7}{$\begin{pmatrix}#1\end{pmatrix}$}}

%Fonts:

\setmainfont[Ligatures=TeX,SmallCapsFont={Latin Modern Roman Caps}]{Georgia}
\setsansfont{Segoe UI}
\setmonofont{Consolas}
\usepackage{unicode-math}
\setmathfont{xits-math.otf}

\usepackage{newunicodechar}
\newunicodechar{Ø}{\emptyset}

\begin{document}
    \section*{Exercise 2}
    
    \def\Rel{\text{\bf Rel}}
    \def\Sets{\text{\bf Sets}}
    \begin{enumerate}[(a)]
        \item $\Rel$ is isomorphic to $\Rel^{op}$ by the ``trivial'' isomorphism $F$:
        \[ F(S) = S, \quad\quad F(R) = \setgen{(b,a)}{(a,b) ∈ R} \]
        
        It is easy to check that this maps identity morphisms to identity morphisms and if $R_1 ⊂ A × B$ and $R_2 ⊂ B × C$ are relations then:
        
        \begin{align*}
                F(R_2 \circ R_1) &= \setgen{(c,a)}{(a,c) ∈ R_2 \circ R_1}\\&= \setgen{(c,a)}{∃b: (a,b) ∈ R_1 ∧ (b,c) ∈ R_2} \\&= \setgen{(c,a)}{∃b: (b,a) ∈ F(R_1) ∧ (c,b) ∈ F(R_2)} \\&= F(R_1) \circ F(R_2) \\&= F(R_2) \circ^{op} F(R_1)
        \end{align*}
        
        It is easy to see that $F$ is self inverse (and the inverse is also a Funktor).
        
        \item $\Sets$ is not isomorphic to $\Sets^{op}$. Assume otherwise, then there are Funktors $F : \Sets → \Sets^{op}$ and $F' : \Sets^{op} → \Sets$ that are inverse to one another.
        In particular $F'(F(Ø)) = Ø$. Assume there is a morphism $f : X → F(Ø)$ in $\Sets^{op}$. Then $F'(f) : F'(X) → Ø$ and we see $F'(X) = Ø$ by the properties of the empty set. Since $F'$ must be monic (``injective'') we conclude $X = F(Ø)$. In otherwords, there must be a magical object $X$ in $\Sets^{op}$ whose only incoming morphisms come from itself. This corresponds to a set whose only outgoing morphisms are to itself. This does not exist.
        \item They cannot be isomorphic since this would correspond to a bijection between a set and its powerset which never exists (well known theorem by Canthor).
    \end{enumerate}
    
    \section*{Exercise 3}
    
    \begin{enumerate}[(a)]
        \item Let $f : A → B$ be an isomorphism in $\Sets$, meaning there is $f^{-1} : B → A$ such that $f' \circ f = id_A$ and $f \circ f' = id_B$. The former tells us that $f$ is injective, the latter tells us that $f$ is surjective. Therefore it is a bijection.\\
        On the other hand, if $f$ is a bijection, then we immediately know that such an $f^{-1}$ exists.
        
        \item Let $f: M_1 → M_2$ be an isomorphism. Since the forgetful functor is a functor, the morphism $f$ is also an isomorphism on the underlying set and therefore a bijection. By definition it is a homomorphism.\\
        Conversely, if $f : M_1 → M_2$ is a homomorphism and bjective, then the inverse function $f^{-1}$ is also a monoid homomorphism since it maps the unit element to the unit element (which is unique) and we have:
        \[ f^{-1}(a \cdot b) = f^{-1}(f(f^{-1}(a)) \cdot f(f^{-1}(b))) = f^{-1}(f(f^{-1}(a) \cdot f^{-1}(b))) = f^{-1}(a) \cdot f^{-1}(b) \]
    \end{enumerate}
\end{document}