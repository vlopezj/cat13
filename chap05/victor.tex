\documentclass{article}

\usepackage{xltxtra}
\usepackage{amsmath}
\usepackage{tikz}
\usepackage{unicode-math}
\usepackage{fontspec}
\setmathfont{xits-math.otf}

\tikzset{node distance=2cm, auto}

\author{Víctor López Juan}
\title{Chapter 5}

\begin{document}

\begin{enumerate}

  \item[8.]

    \begin{itemize}
      \item Composition is well-well defined.
        

      \begin{itemize}
        \item The arrow $\vert f \vert^\star (U_g) \rightarrowtail U_f$
          is monic, because the diagram is a pullback.

        \item $\vert f \vert ^\star (U_g)$ is a subobject of A.

      \end{itemize}


      \item There must be identity arrows for every object.

        \begin{itemize}
          \item For every object, there is an identity arrow.

            \begin{tikzpicture}
              \node (A) {$A$};
              \node (A2) [right of=A] {$A$};
              \node (A3) [below of=A] {$A$};
    
              \draw [->] (A) to node {$id_A$} (A2);
              \draw [>->] (A) to node {$id_A$} (A3);
            \end{tikzpicture}
    

            \begin{tikzpicture}
              
            \end{tikzpicture}
            
            Pullbacks are unique. Therefore, any object which makes
            the diagram commute and is universal with this property
            is the pullback.
            
            \begin{tikzpicture}
              \node (A) {$A$};
              \node (A2) [right of=A] {$A$};
              \node (A3) [below of=A] {$A$};

              \node (Uf) [above of=A2] {$U_f$};
              \node (Uf2) [above of=A] {$U_f$};
              \node (B)  [right of=Uf] {$B$};
              
    
              \draw [->] (A) to node {$id_A$} (A2);
              \draw [>->] (A) to node {$id_A$} (A3);
              \draw [>->] (Uf) to node {$m$} (A2);
              \draw [>->] (Uf2) to node {$m$} (A);
              \draw [->] (Uf) to node {$\vert f \vert$} (B);
              \draw [->] (Uf2) to node {$id_{U_f}$} (Uf);
            \end{tikzpicture}


            \begin{tikzpicture}

              \node (Uf) {$U_f$};
              \node (A)  [below of=Uf] {$A$};
              \node (Uf2) [above of=Uf] {$U_f$};
              
              \node (B)  [right of=Uf] {$B$};
              \node (B2) [above of=B] {$B$};
              \node (B3) [right of=B2] {$B$};

    
              \draw [->] (B2) to node {$id_B$} (B3);
              \draw [>->] (B2) to node {$id_B$} (B);
              \draw [>->] (Uf) to node {$m$} (A);
              \draw [->] (Uf) to node {$\vert f \vert$} (B);
              \draw [->] (Uf2) to node {$\vert f \vert$} (B2);
              \draw [>->] (Uf2) to node {$id_{U_f}$} (Uf);
            \end{tikzpicture}
            % TODO: Check it is a pullback
            Easy to check that $U_f$ is a pullback of both diagrams.

            Therefore, identity arrows are both the right and left
            identity for composition

        \item Composition is associative.

           \begin{tikzpicture}

             \node (A) {$A$};
             \node (Uf) [above of=A] {$U_f$};

             \node (B) [right of=Uf] {$B$};
             \node (Ug) [above of=B] {$U_g$};

             \node (C) [right of=Ug] {$C$};
             \node (Uh) [above of=C] {$U_h$};

             \node (D) [right of=Uh] {$D$};

             \draw [>->] (Uf) to node {$m_1$} (A);
             \draw [>->] (Ug) to node {$m_2$} (B);
             \draw [>->] (Uh) to node {$m_3$} (C);

             \draw [->]  (Uf) to node {$\vert f \vert$} (B);
             \draw [->]  (Ug) to node {$\vert g \vert$} (C);
             \draw [->]  (Uh) to node {$\vert h \vert$} (D);
          \end{tikzpicture}
           
          \begin{tikzpicture}

             \node (A) {$A$};
             \node (Uf) [above of=A] {$U_f$};

             \node (B) [right of=Uf] {$B$};
             \node (Ug) [above of=B] {$U_g$};
             \node (Ug2) [above of=Uf] {$U_{f \circ g}$};

             \node (C) [right of=Ug] {$C$};
             \node (Uh) [above of=C] {$U_h$};
             
             \node (Uh2) [above of=Ug2] {$\vert f \circ g \vert U_h$};

             \node (D) [right of=Uh] {$D$};

             \draw [>->] (Uf) to node {$m_1$} (A);
             \draw [>->] (Ug) to node {$m_2$} (B);
             \draw [>->] (Ug2) to node {$m_2^\prime$} (Uf);
             \draw [>->] (Uh) to node {$m_3$} (C);

             \draw [->]  (Ug2) to node {$\bar{\vert f \vert}$} (Ug);
             
             \draw [->]  (Uf) to node {$\vert f \vert$} (B);
             \draw [->]  (Ug) to node {$\vert g \vert$} (C);
             \draw [->]  (Uh) to node {$\vert h \vert$} (D);

             \draw [>->] (Uh2) to node {$m_3^\prime$} (Ug2);
             \draw [->] (Uh2) to node {} (Uh);
          \end{tikzpicture}
          
           \begin{tikzpicture}

             \node (A) {$A$};
             \node (Uf) [above of=A] {$U_f$};

             \node (B) [right of=Uf] {$B$};
             \node (Ug) [above of=B] {$U_g$};

             \node (C) [right of=Ug] {$C$};
             \node (Uh) [above of=C] {$U_h$};

             \node (Uh2) [above of=Ug] {$U_{g \circ h}$};
             
             \node (Ug2) [left of=Uh2] {$\vert f \vert^\star U_{g \circ h}$};

             \node (D) [right of=Uh] {$D$};

             \draw [>->] (Uf) to node {$m_1$} (A);
             \draw [>->] (Ug) to node {$m_2$} (B);
             \draw [>->] (Uh) to node {$m_3$} (C);

             \draw [>->] (Uh2) to node {$m_3^\prime$} (Ug);
             
             \draw [>->] (Ug2) to node {} (Uf);
             
             \draw [->]  (Uf) to node {$\vert f \vert$} (B);
             \draw [->]  (Ug) to node {$\vert g \vert$} (C);
             \draw [->]  (Uh) to node {$\vert h \vert$} (D);

             \draw [->]  (Uh2) to node {$\bar{\vert g \vert}$} (Uh);


             \draw [->]  (Ug2) to node {} (Uh2);
             
          \end{tikzpicture}
           
           \begin{tikzpicture}

             \node (A) {$A$};
             \node (Uf) [above of=A] {$U_f$};

             \node (B) [right of=Uf] {$B$};
             \node (Ug) [above of=B] {$U_g$};

             \node (C) [right of=Ug] {$C$};
             \node (Uh) [above of=C] {$U_h$};

             \node (Ugh) [above of=Ug] {$U_{g \circ h}$};
             \node (Ufg) [above of=Uf] {$U_{f \circ g}$};
             
             \node (Ufgh) [above of=Ufg] {$U_{f \circ g \circ h}$};


             \draw [->]  (Ufg) to node {$\bar{\vert f \vert}$} (Ug);
             \draw [->]  (Ugh) to node {$\bar{\vert g \vert}$} (Uh);
             \draw [->]  (Ufgh) to node {} (Ugh);

             \node (D) [right of=Uh] {$D$};

             \draw [>->] (Uf) to node {$m_1$} (A);
             \draw [>->] (Ug) to node {$m_2$} (B);
             \draw [>->] (Uh) to node {$m_3$} (C);

             \draw [>->] (Ufg) to node {} (Uf);
             \draw [>->] (Ugh) to node {} (Ug);
             \draw [>->] (Ufgh) to node {} (Ufg);

             \draw [->]  (Uf) to node {$\vert f \vert$} (B);
             \draw [->]  (Ug) to node {$\vert g \vert$} (C);
             \draw [->]  (Uh) to node {$\vert h \vert$} (D);
          \end{tikzpicture}


          By the two pullback lemma, $U_{f \circ g \circ h}$ is both a pullback
          of, and, :

          % TODO: Explain

          Pullbacks are unique up to isomorphism. Therefore:

          $$ \vert f \vert^\star U_{g\circ h} = U_{f \circ g \circ h} = \vert f \circ g \vert^\star (U_h)$$

        \end{itemize}
        
   \end{itemize}

   \item[13.]

     First, let's define the limits in terms of products and equalizers
     in Set.

     \begin{itemize}
       \item
         
         $$M = \lim_{\righttarrow} M_i = \nicefrac{\coprod M_i}{~}$$

         Where $~$ is defined as the relation that makes each of the
         following diagrams comute:

         $$ in_i = in_{i+1} \comp g_i$$

         %% TODO

       \item
         
         
         $$M = \lim_{\lefttarrow} M_i = {\prod M_i} \vert_P$$

         Where $P$ is the conjuction of the following:

         $P(m) \equiv \bigwedge_i \pi_{i+1}(m) = \pi_{i}( g_{i}(m) )$

         If $\pi_1(m)$ is fixed, then the values of $\pi_{i+1}(m)$ are
         all determined by $P$. Essentially, $M = M_o$.

       \item

         $$N = \lim_{\righttarrow} N_i = \nicefrac{\coprod N_i}{~}$$

         $$ in_{i+1} = in_{i} \comp g_i$$
         
         Every element $n \in N_i$ is a member of the equivalence
         class of $g_{i,0}(n) \in N_0$, and each of those elements
         has its own equivalence class in the quotient.
         Therefore, $N = N_0$.

       \item  

         $$M = \lim_{\lefttarrow} M_i = {\prod M_i} \vert_P$$
         
         $P(m) \equiv \bigwedge_i \pi_{i}(m) = g_{i}(\pi_{i+1}(m) )$

         $M$ is defined as an equalizer.

     \begin{enumerate}

       \item[a.]

         Cases 2 and 3 are trivial.

         In cases 1 and 4, the group operation is defined pointwise,
         so commutativity and inverse is inherited.
         
     All constructions preserve being abelian, because, in all of them,
     the product is defined with regards to the underlying monoids,
     which are all abelian groups.

     \item[b.]

         Cases 2 and 3 are trivial, because they are finite groups.

         In case 1, all elements of $M$ are also elements of one of the
         $M_i$, which is finite. Therefore, they all have finite order.

         In case 4, this is, generally, not the case.

         Take the sequence $M_i = (\mathbb{Z}_{2^i},+)$.

         The infinite tuple $(1, …, 1) \in M$. However, for every $k$,
         $k·(1, …, 1) \neq 0$.
         
         
     \end{enumerate}

    \end{enumerate}

\end{enumerate}
\end{document}
  
