%use lualatex/xelatex to compile this
\documentclass{scrartcl}
\usepackage{polyglossia}
\setdefaultlanguage{english}

\usepackage{marvosym}% für den Smiley :-)
\def\contradiction{\quad\text{\Large\Lightning}}
\usepackage{graphicx}
\usepackage{float}
\usepackage{listings}
\usepackage{color}
\usepackage{enumerate}

\lstset{%
  literate={ö}{{\"o}}1%
           {ä}{{\"a}}1%
           {ü}{{\"u}}1%
           {Ö}{{\"O}}1%
           {Ä}{{\"A}}1%
           {Ü}{{\"U}}1%
           {ß}{{\ss}}1%
}

\definecolor{bggray}{rgb}{0.90,0.90,0.90}
\definecolor{dkgreen}{rgb}{0.00,0.50,0.00}
\definecolor{mauve}{rgb}{0.50,0.00,0.30}
\definecolor{darkGray}{rgb}{0.40,0.40,0.40}
\lstset{%
  language=C++,                   % the language of the code
  basicstyle=\small\ttfamily,
  numbers=none,                   % where to put the line-numbers
  backgroundcolor=\color{bggray}, % choose the background color. You must add \usepackage{color}
  showspaces=false,               % show spaces adding particular underscores
  showstringspaces=false,         % underline spaces within strings
  showtabs=false,                 % show tabs within strings adding particular underscores
  frame=single,                   % adds a frame around the code
  tabsize=4,                      % sets default tabsize to 2 spaces
  breaklines=true,                % sets automatic line breaking
  breakatwhitespace=true,         % sets if automatic breaks should only happen at whitespace
  title=\lstname,                 % show the filename of files included with \lstinputlisting;
                                  % also try caption instead of title
  keywordstyle=\color{blue},      % keyword style
  commentstyle=\color{dkgreen},   % comment style
  stringstyle=\color{mauve},      % string literal style
  escapeinside={\%*}{*)},         % if you want to add a comment within your code
}

\usepackage{amsmath,amsfonts,amsthm,amssymb,bbm,dsfont,mathrsfs}
\delimitershortfall=-3pt %makes parentheses grow larger automagically

\DeclareMathOperator*{\argmin}{argmin}
\DeclareMathOperator{\argmax}{argmax}
\DeclareMathOperator*{\cupdot}{\stackrel{{}_\bullet}{\cup}}
\DeclareMathOperator*{\bigcupdot}{\stackrel{{}_\bullet}{\bigcup}}
\def\LHS{&\phantom{{}=}} % First line of align environment with no relation symbol should still be align with the other lines. This is a dirty fix.
\usepackage{txfonts}

%Komma und Punktabstände anpassen:
%\mathcode`,="013B %nervt, außerdem braucht niemand Kommazahlen.
%\mathcode`.="613A

\author{Stefan Walzer}

%MENGEN Symbole
\newcommand{\R}{\ensuremath{\mathbb R}}
\newcommand{\N}{\ensuremath{\mathbb N}}
\newcommand{\Z}{\ensuremath{\mathbb Z}}
\newcommand{\Q}{\ensuremath{\mathbb Q}}
\newcommand{\C}{\ensuremath{\mathbb C}}
\newcommand{\K}{\ensuremath{\mathbb K}}

%Quantoren
\newcommand{\E}{\ensuremath{\exists}}
\newcommand{\A}{\ensuremath{\forall}}

%Pfeile
\newcommand{\LA}{\ensuremath{\Leftarrow}}
\newcommand{\RA}{\ensuremath{\Rightarrow}}
\newcommand{\la}{\ensuremath{\leftarrow}}
\newcommand{\ra}{\ensuremath{\rightarrow}}
\newcommand{\LRA}{\ensuremath{\Leftrightarrow}}
\newcommand{\lra}{\ensuremath{\leftrightarrow}}
\newcommand{\LLRA}{\ensuremath{\Longleftrightarrow}}
\newcommand{\llra}{\ensuremath{\longleftrightarrow}}

%spitze Klammern
\newcommand{\lan}{\ensuremath{\langle}}
\newcommand{\ran}{\ensuremath{\rangle}}


%Symbole
\newcommand{\veps}{\ensuremath{\varepsilon}}

%Verknüpfungen
\newcommand{\AND}{\wedge} %und
\newcommand{\OR}{\vee} %oder
\newcommand{\UN}{\cup} %union
\newcommand{\IS}{\cap} %intersection

%sonstiges
\def\abs#1{\left|#1\right|}
\def\norm#1{\lVert#1\rVert}
\def\floor#1{\left\lfloor#1\right\rfloor}
\def\ceil#1{\left\lceil#1\right\rceil}
\def\enquote#1{\glqq #1 \grqq}

\def\powset#1{\mathscr{P}\left(#1\right)}
\def\set#1{\left\{#1\right\}}
\def\setgen#1#2{\left\{#1\;\middle|\;#2\right\}}

%vector
\newcommand{\vect}[1]{\begin{pmatrix}#1\end{pmatrix}}
\newcommand{\svect}[1]{\scalebox{0.7}{$\begin{pmatrix}#1\end{pmatrix}$}}

%Fonts:

\setmainfont[Ligatures=TeX,SmallCapsFont={Latin Modern Roman Caps}]{Georgia}
\setsansfont{Segoe UI}
\setmonofont{Consolas}
\usepackage{unicode-math}
\setmathfont{xits-math.otf}

\usepackage{newunicodechar}
\usepackage{tikz}
\newunicodechar{Ø}{\emptyset}
\tikzset{node distance=3.5cm, auto}

\begin{document}
    \section*{Exercise 1}
    
    Let $f × g : P → X$ be the product of $f$ and $g$ in the slice category (note: This is not the normal product!), with projections $π_f : P → A$ and $π_g : P → B$ such that $f \circ π_f = g \circ π_g = f × g$ (this is what it means to be a morphism in the slice category).
    
    Because of this equality we immediately know that $π_f$ and $π_g$ factorise uniquely over $A ×_{X} B$ via a morphism $i_1 : P → A ×_{X} B$ (this uses the property of the pullback).
    
    On the other hand $p_1$ and $p_2$ are morphisms from $f \circ p_1 = g \circ p_2$ to $f$ and $g$ in the slice category respectively, therefore they factorise uniquely over the product $f × g$ via some unique morphism $i_2: A ×_{X} B → P$. By uniqueness, we have that $i_1$ and $i_2$ must be inverse to one another and therefore pullbacks are exactly products in the slice category.
    
    \begin{figure}[H]\centering\begin{tikzpicture}
      \node  (X) {$X$} ;
      \node  (A) [left of=X] {$A$} ;
      \node  (B) [right of=X] {$B$} ;
      \node  (P) [above of=X] {$P$} ;
      \node  (AB) [above of=P] {$A ×_X B$} ;
      
      \draw[->] (A) to node {$f$} (X);
      \draw[->] (B) to node {$g$} (X);
      \draw[->] (P) to node {$π_f$} (A);
      \draw[->] (P) to node {$π_g$} (B);
      \draw[->] (P) to node {$f × g$} (X);
      \draw[->] (AB) to node {$p_1$} (A);
      \draw[->] (AB) to node {$p_2$} (B);
    \end{tikzpicture}
    \end{figure}
    
    \section*{Exercise 7}
    
    Let $L$ be the limit of $D$ with (with morphisms $l_i : L → C_i$).\\
    Let $L'$ be the limit of $Hom(C, -) \circ D$ (with morphisms $l'_i : L' → Hom(C, C_i)$).\\

    We have to show that $Hom(C, L) \cong L'$.
    
    The unique morphism from $Hom(C,L)$ to $L'$ exists since $L'$ is a limit and mapping a cone via a functor yields a cone. The other direction is the interesting one:
    
    Let $x ∈ L'$ be an element of $L'$ (note that we are in {\bf Sets}). The morphisms $l'_i$ map this $x$ to morphisms $l'_i(x) : C → C_i$. Because $L$ was a limit over $D$, we know that there is a factorisation through $L$, i.e. there exists a unique $ξ : C → L$ such that $l'_i(x) = l_i \circ ξ$. In other words, we indentified the unique element of $Hom(C,L)$ that behaves under $l_i$ in the same way that $x$ behaves under $l'_i$. Since $x ∈ L'$ was chosen arbitrarily, we conclude that there is a unique map from $L' → Hom(C,L)$ that makes everything commute. This completes the contruction of the isomorphism and $Hom(C,L)$ is therefore also a limit of $Hom(C,-) \circ D$.
\end{document}